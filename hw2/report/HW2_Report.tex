%%%%%%%%%%%%%%%%%%%%%%%%%%%%%%%%%%%%%%%%%
% Stylish Curriculum Vitae
% LaTeX Template
% Version 1.1 (September 10, 2021)
%
% This template originates from:
% https://www.LaTeXTemplates.com
%
% Authors:
% Stefano (https://www.kindoblue.nl)
% Vel (vel@LaTeXTemplates.com)
%
% License:
% CC BY-NC-SA 4.0 (https://creativecommons.org/licenses/by-nc-sa/4.0/)
%
%%%%%%%%%%%%%%%%%%%%%%%%%%%%%%%%%%%%%%%%%
% !TEX program = xelatex
\documentclass[a4paper, oneside, final, 12pt]{scrartcl} % Paper options using the scrartcl class

\usepackage{fontspec} % for other font
\usepackage{xeCJK} % for chinese font
\usepackage{hyperref} % for hyper web link
\usepackage{multirow} % for tabular table in learning progress
\usepackage{graphicx} % for image insersion
\usepackage[export]{adjustbox} % for image frame
\usepackage{setspace}
\usepackage{array}
% Define typographic struts, as suggested by Claudio Beccari
%   in an article in TeX and TUG News, Vol. 2, 1993.
\usepackage{mathptmx}
\usepackage{scrlayer-scrpage} % Provides headers and footers configuration
\usepackage{titlesec} % Allows creating custom \section's
\usepackage{marvosym} % Allows the use of symbols
\usepackage{tabularx,colortbl} % Advanced table configurations
% \usepackage{ebgaramond} % Use the EB Garamond font
\usepackage{microtype} % To enable letterspacing
\usepackage{pdfpages} % for showing pdf
\usepackage{pdflscape}
\usepackage{enumitem}
\usepackage{subcaption}
\usepackage{listings}   % highlight the python code
\usepackage{xcolor}
\usepackage{multirow}
\usepackage{cite} %Imports biblatex package
\usepackage[ruled,linesnumbered]{algorithm2e}
\newcommand\mycommfont[1]{\normalsize\ttfamily\textcolor{blue}{#1}}
\SetCommentSty{mycommfont}
% \usepackage[backend=bibtex,bibencoding=ascii,style=authoryear,sorting=none]{bibtex}
% \addbibresource{reference.bib}
% setup the margin
\usepackage[top=1cm, bottom=1cm, right=2cm, left=2cm]{geometry}

% set the style of listing code
\definecolor{codegreen}{rgb}{0,0.6,0}
\definecolor{codegray}{rgb}{0.5,0.5,0.5}
\definecolor{codepurple}{rgb}{0.58,0,0.82}
\definecolor{backcolour}{rgb}{0.95,0.95,0.92}

\lstdefinestyle{mystyle}{
    backgroundcolor=\color{backcolour},   
    commentstyle=\color{codegreen},
    keywordstyle=\color{magenta},
    numberstyle=\tiny\color{codegray},
    stringstyle=\color{codepurple},
    basicstyle=\ttfamily\footnotesize,
    breakatwhitespace=true,         
    breaklines=true,                 
    captionpos=b,                    
    keepspaces=true,                 
    numbers=left,                    
    numbersep=5pt,                  
    showspaces=false,                
    showstringspaces=false,
    showtabs=false,                  
    tabsize=2
}

\lstset{style=mystyle}

% set chinese and english font
\setmainfont{Times New Roman}
\setCJKmainfont[AutoFakeBold=true, AutoFakeSlant=true]{標楷體}

\titleformat{\section}{\Large\raggedright\bfseries}{}{0em}{}[\titlerule] % Section formatting
\titleformat{\subsection}{\large\raggedright\bfseries}{}{0em}{}
\titleformat{\subsubsection}{\normalsize\raggedright\bfseries}{}{0em}{}

% \pagestyle{scrheadings} % Print the headers and footers on all pages

% enable bold and slant chinese font
% \xeCJKsetup{AutoFakeBold=true, AutoFakeSlant=true}

% set the space at the front of paragraph
\setlength{\parindent}{2em}

% disable page number
\pagenumbering{gobble}

\newcommand{\gray}{\rowcolor[gray]{.90}} % Custom highlighting for the work experience and education sections
\newcommand{\Tstrut}{\rule{0pt}{2.6ex}}         % = `top' strut
\newcommand{\Bstrut}{\rule[-0.9ex]{0pt}{0pt}}   % = `bottom' strut
\newcommand{\Tstruth}{\rule{0pt}{4ex}}         % = `top' strut for header
\newcommand{\Bstruth}{\rule[-2.5ex]{0pt}{0pt}}   % = `bottom' strut for header

%----------------------------------------------------------------------------------------
%	FOOTER SECTION
%----------------------------------------------------------------------------------------

% \renewcommand{\headfont}{\normalfont\rmfamily\scshape} % Font settings for footer

% \cofoot{
% \fontsize{12.5}{17}\selectfont % Letter spacing and font size

% \textls[150]{123 Broadway {\large\textperiodcentered} City {\large\textperiodcentered} Country 12345}\\ % Your mailing address
% {\Large\Letter} \textls[150]{john@smith.com \ {\Large\Telefon} (000) 111-1111} % Your email address and phone number
% }

%----------------------------------------------------------------------------------------
\begin{document}

%----------------------------------------------------------------------------------------
%	HEADER SECTION
%----------------------------------------------------------------------------------------


\begin{center}
    {\fontsize{18}{30}\textbf{Data Mining Assignment 2 \\ Classification}}
\end{center}

\begin{center}
  Bo-Han Chen (陳柏翰) \\
  Student ID:312551074 \\
  bhchen312551074.cs12@nycu.edu.tw
\end{center}

\section{Experiment Environment \& Usage}

\begingroup
\raggedright

\subsection{Environment}

\begin{itemize}
  \item OS: Windows 10 22H2
  \item Hardware: Intel(R) Xeon(R) Gold 6126 CPU @ 2.60GHz
  \item Python 3.9.17
  \item The computation time is recorded by \emph{time.process\_time()} function
\end{itemize}

\subsection{Usage}

The command for executing the program of step2 \& 3 is shown as follows:

\begin{lstlisting}[language=bash]
  # step2
  python ./apriori.py -f [Input File] -t [Task] -s [Min Support]
  # step3
  python ./myEclat.py -f [Input File] -s [Min Support]
\end{lstlisting}

I wrote a script for running the association rule mining program,
which can run the algorithm with all task/support/dataset options,
and the execution time will be recorded in the log file named \emph{result.log}.
The experiment result in this report is 
based on the \emph{result.log} generated by the script. \\
Take the script of step2 for example, the script 
\emph{run.sh} is shown as follows:

\section{Step2: Apriori Algorithm}

\subsection{Task1: Mining Frequent Itemsets}

In this part, I add two functions \emph{writeTask1\_1} and \emph{writeTask1\_2} to
write the frequent itemsets to the txt file based on the original Apriori algorithm. 
The code is shown as follows:

\begin{lstlisting}[language=Python]
  def runApriori_1(data_iter, case, minSupport):
    itemSet, transactionList = getItemSetTransactionList(data_iter)

    freqSet = defaultdict(int)
    largeSet = dict()
    # initialize the number of candidate itemset before and after pruning
    canNumSetBf = [len(itemSet)]
    canNumSetAf = []

    oneCSet= returnItemsWithMinSupport(itemSet, transactionList, minSupport, freqSet)
    canNumSetAf.append(len(oneCSet))
    
    currentLSet = oneCSet
    k = 2
    while currentLSet != set([]):    
        largeSet[k - 1] = currentLSet
        currentLSet = joinSet(currentLSet, k)
        # get the number of candidate itemset before pruning
        canNumSetBf.append(len(currentLSet))
        currentCSet= returnItemsWithMinSupport(
            currentLSet, transactionList, minSupport, freqSet
        )
        # get the number of candidate itemset after pruning
        canNumSetAf.append(len(currentCSet))
        currentLSet = currentCSet
        k = k + 1
    ...
    # write the frequent itemsets and number of candidate to file
    writeTask1_1(toRetItems, case, minSupport)
    writeTask1_2(canNumSetBf, canNumSetAf, case, minSupport)
\end{lstlisting}

The computation time of task1 is shown as follows 
(concluded from the \emph{result.log} file):

\begin{table}[ht]
  \centering
    \begin{tabular}{|*{3}{c|}}
        \hline
    Dataset    & Minimum Support (\%)  & Computation Time (sec)  \\
        \hline
    \multirow[t]{3}{*}{}           
                & \multirow[t]{3}{*}{}0.2
                            & 143.79 \\  \cline{2-3}
                A& 0.5          & 6.72 \\  \cline{2-3}
                & 1.0          & 2.79 \\  \cline{1-3}         
                & \multirow[t]{3}{*}{}0.15
                            & 6861.15 \\  \cline{2-3}
                B& 0.2          & 3823.43 \\  \cline{2-3}
                & 0.5          & 1111.96 \\  \cline{1-3}
                & \multirow[t]{3}{*}{}1.0
                            & 6074.08 \\  \cline{2-3}
                C& 2.0          & 1994.86 \\  \cline{2-3}
                & 3.0          & 729.53 \\ 
        \hline
    \end{tabular}
  \caption{Computation Time of Task1}
\end{table}

The computation time of task2 and the comparison with task1 is shown as follows:

\begin{table}[ht]
  \centering
    \begin{tabular}{|*{4}{c|}}
        \hline
    Dataset & Minimum Support (\%)  & Computation Time (sec) & Ratio of Computation Time (\%)  \\
        \hline
    \multirow[t]{3}{*}{}           
                & \multirow[t]{3}{*}{}0.2
                            & 157.117 & 109.26\% \\  \cline{2-4}
                A& 0.5          & 6.65 & 98.95\% \\  \cline{2-4}
                & 1.0          & 2.70 & 96.77\% \\  \cline{1-4}         
                & \multirow[t]{3}{*}{}0.15
                            & 7094.64 & 103.40\% \\  \cline{2-4}
                B& 0.2          & 3730.72 & 97.57\% \\  \cline{2-4}
                & 0.5          & 1137.05 & 102.25\% \\  \cline{1-4}
                & \multirow[t]{3}{*}{}1.0
                            & 6007.42 & 98.90\% \\  \cline{2-4}
                C& 2.0          & 1962.21 & 98.36\% \\  \cline{2-4}
                & 3.0          & 717.23 & 98.31\% \\ 
        \hline
    \end{tabular}
  \caption{Computation Time of Task2}
\end{table}

\section{Step3: Eclat Algorithm}

\subsection{Program Flow}

The pueudocode of Eclat algorithm is shown as follows:

\begin{algorithm}
  \caption{My Eclat Algorithm Overview}
  % \SetKwData{Left}{left}\SetKwData{This}{this}\SetKwData{Up}{up}
  % \SetKwFunction{Union}{Union}
  \SetKwFunction{EclatRecursive}{EclatRecursive}
  \SetKwInOut{Input}{input}\SetKwInOut{Output}{output}

  \Input{Transaction list $T$, minimum support $Sup_{min}$}
  \Output{Frequent itemsets $F$}
  \BlankLine
  \tcp{build the vertical database $VDB$ from $T$}
  \For{$tran$ in $T$}{
    \For{$item$ in $tran$}{
      append $tran$ to $VDB[item]$
    }
  }
  \tcp{get the frequent 1-itemsets $F_1$ from $VDB$}
  \For{$item$ in $VDB$}{
    \If{$|VDB[item]| \geq Sup_{min}$}{
      append $item$ to $F_1$
    }
  }
  \tcp{mine the frequent itemsets recursively}
  $F = []$ \\
  \For{$item$ in $F_1$}{
    \EclatRecursive{$item$, $VDB[item]$, $idx(item)$}
  }
\end{algorithm}

\begin{algorithm}
  \caption{EclatRecursive Function}
  \SetKwInOut{Input}{input}\SetKwInOut{Output}{output}

  \Input{frequent itemset $i$, tid set $SET_{i}$, index of $i$'s last item $IDX_{i}$}
  \BlankLine
  \For{$j\leftarrow IDX_{i} + 1$ \KwTo $|F_1|$}{
    $SET_{ij} = SET_i \cap VDB[j]$ \\
    \If{$|SET_{ij}| \geq Sup_{min}$}{
      $i_{new} = i \cup F_1[j]$ \\
      append $i_{new}$ to $F$ \\
      \EclatRecursive{$i_{new}$, $SET_{ij}$, $j$}
    }
  }
\end{algorithm}

\newpage

reftest\cite{zaki1997new}

\endgroup

\bibliographystyle{unsrt} % We choose the "plain" reference style
\bibliography{reference} % Entries are in the "references.bib" file

\end{document}